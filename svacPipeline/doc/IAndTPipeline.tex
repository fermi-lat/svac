\documentclass{article}
\usepackage[dvips]{graphics}

\begin{document}

\title{The I\&T Pipieline}
\author{Warren Focke - I\&T/SVAC}
\date{2004 December 15}
\maketitle


\section{Introduction}
\label{intro-sec}

This is obviously very preliminary, I mostly just wanted to get the graphs
out.


\section{Logical Flow}
\label{logic-sec}

Figure \ref{desired-fig} depicts dependency flow in the I\&T pipeline task set.
This is an idelaized version of what we'd like, the actual mapping of this
structures onto the capabilities of the current GINO implementation will be
discussed in Section \ref{implement-sec}.

\begin{figure}
\label{desired-fig}
\caption{Ideal}
\includegraphics{desired-EM2-v1r0.eps}

{TPs are in boxes. Blue boxes are reprocessing entry points.  Datasets are
ellipses. Non-file dependencies are triangles.  Solid lines show input/output
relationships of TPs and DSs.}

\end{figure}

There are 15 task processes (TPs), 8 of which are entry points.  11 Datasets
and 1 non-file dependency (creation of a database record for the run) are
shown.  Several datasets internal to a single TP are not shown here.




\section{Implementation}
\label{implement-sec}

Figure \ref{actual-fig} shows how we mapped the logical dependency flow onto
the capabilities of GINO.

\begin{figure}
\label{actual-fig}
\caption{Actual}
\includegraphics{actual-EM2-v1r0.eps}

{Each entry point represents a task.  Dotted lines show the sequence of TPs
within a task.  Dashed lines show launching of tasks by TPs.  TPs and datasets
are labeled taskName\_tpName and taskName\_dsName (version numbers removed
from taskName) so they can be identified with the matching entities in the
GINO configration.}

\end{figure}

There are more TPs (22 here), since it takes a TP to launch a task.  Under a
future version of GINO which understood parallel execution, branching
dependency flow, and multiple entry points, all of this stuff would be in 2
tasks (instead of 8 now), and all but 1 of these "chaining" TPs would be
eliminated.

There appear to be more datasets.  This is beacuse the DSs seen by a task
include the task name, so the TPs that launch a task must make copies of or
links to Dss that are passed on to the next task.  Thus, digi\_digi\_root,
recon\_digi\_root, digiReport\_digi\_root, svac\_digi\_root, and
reconReport\_digi\_root are all the same data, but it must appear with all those
names to work with GINO's naming convention.

There are more edges.  These represent extra information that must be
maintained by the user, as the ordering of TPs within a task must be
explicitly specified to GINO, and launching of other tasks by TPs is done ``by hand.''

The dependency on the eLog DB is not explicitly present - it is guaranteed by
the ordering of the tasks.

I have in mind an alternate (harder, of course) approach to launching tasks
that would reduce latency waiting for tasks to run and simplify this graph
\emph{a bit}.


\end{document}
