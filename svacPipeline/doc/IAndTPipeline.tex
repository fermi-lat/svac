\documentclass{article}
\usepackage[dvips]{graphicx}

\begin{document}

\title{The I\&T Pipieline}
\author{Warren Focke - I\&T/SVAC}
\date{2004 December 15}
\maketitle

\def\figref#1{Figure~\ref{#1}}
\def\secref#1{Section~\ref{#1}}

\section{Introduction}
\label{intro-sec}

This is obviously very preliminary, I mostly just wanted to get the graphs
out.

\section{Products}
\label{prod-sec}

The main products of the pipeline are 1 FITS and 4 ROOT files:

\begin{itemize}

\item LDF
\begin{itemize}
\item Provided by Online
\item Raw data in electronics space (topological)
\item Raw (.ldf) and FITS (.fits) flavors
\item FITS is the official product
\item Opaque (but FITS can separate events)
\end{itemize}

\item Digi
\begin{itemize}
\item Raw data in detector space (geometrical)
\item Tree
\item http://confluence.slac.stanford.edu/display/WB/digiRootData
\end{itemize}

\item Recon
\begin{itemize}
\item Reconstructed data, plus details of recon process
\item Tree
\item http://confluence.slac.stanford.edu/display/WB/reconRootData
\end{itemize}

\item Merit
\begin{itemize}
\item High-level summary of reconstructed data
\item Tuple
\item http://confluence.slac.stanford.edu/display/WB/Merit+Ntuple
\end{itemize}

\item SVAC
\begin{itemize}
\item High and low-level data
\item Tuple, plus fixed-size arrays
\item http://www.slac.stanford.edu/exp/glast/ground/software/SVAC/EngineeringModelRoot/main.html (Doc)
\item http://www-glast.slac.stanford.edu/IntegrationTest/SVAC/Instrument\_Analysis/Talks/Tuesday/xin\_svac.pdf (Rationale \& tutorial)
\end{itemize}

\end{itemize}

\section{Logical Flow of I\&T Tasks}
\label{logic-sec}

{\figref{desired-fig}} depicts dependency flow in the I\&T pipeline task set.
This is an idelaized version of what we'd like, the actual mapping of this
structures onto the capabilities of the current GINO implementation will be
discussed in {\secref{implement-sec}}.

\begin{figure}
\caption{Ideal}

\includegraphics[bb=135 40 677 415,width=7.5in,height=6.5in]{desired-EM2-v1r0.eps}

{TPs are in boxes. Shaded boxes are reprocessing entry points.  Datasets are
ellipses. Non-file dependencies are triangles.  Solid lines show input/output
relationships of TPs and DSs.}

\label{desired-fig}
\end{figure}

Don't want things to break because soemthing that they didn't actually depend
on failed.

Want multiple entry points for reprocessing.

There are 15 task processes (TPs), 8 of which are entry points.  11 Datasets
and 1 non-file dependency (creation of a database record for the run) are
shown.  Several datasets internal to a single TP are not shown here.

\section{GINO}
\label{gino-sec}

GINO is the infrastructure upon which the I\&T pipeline is implemented.

A Task in GINO consists of a collection of Processes (TPs) and Datasets (DSs),
along with dependency relations between them.  To actually perform the Task, a
Run is created with a unique Run Name (in practice, an integer).

The TPs are excecuted sequentially, in an order specified by the user when the
Task is set up.  Processing always starts with the first TP in the Task.  If a
TP fails, those later in the sequence will not be attempted.

Datasets are are written by at most one TP.  They may be read-only, in which
case they are assumed to exist before a Run of the Task is created.  They may
be read by any number of TPs, including none.  The filenames of DSs are
assembled by GINO from the Task Name, Run Name, and several other fields
supplied by the user at Task creation.

\section{Implementation of the I\&T Pipieline}
\label{implement-sec}

{\figref{actual-fig}} shows how we mapped the logical dependency flow onto
the capabilities of GINO.

\begin{figure}
\caption{Actual}

\includegraphics[bb=120 38 650 502,width=7.5in,height=6.5in]{actual-EM2-v1r0.eps}

{Each entry point represents a task.  Hexagons show symbolic links. Dashed
lines show the sequence of TPs within a task.  Dotted lines show launching of
tasks by TPs and creation of symbolic links, both of which happen witho GINO's
knowledge.  TPs and datasets are labeled taskName:tpName and taskName:dsName
(version numbers removed from taskName) so they can be identified with the
matching entities in the GINO configration.}

\label{actual-fig}
\end{figure}

There are more TPs (22 here), since it takes a TP to launch a task.  Under a
future version of GINO which understood parallel execution, branching
dependency flow, and multiple entry points, all of this stuff would be in 2
tasks (instead of 8 now), and all but 1 of these "chaining" TPs would be
eliminated.  

Note that each TP involves at least two scripts.  Setting this up required
filling out over 400 fields trough a web interface.  They had to be entered
without error, as they could not be edited once entered.  That web interface
is now deprecated, configuration is done with XML files which can be edited
offline or machine-generated.

There appear to be more datasets.  This is beacuse the DSs seen by a task
include the task name, so the TPs that launch a task must make copies of or
links to DSs that are passed on to the next task.  Thus, digi\_digi\_root is a
real file, while recon\_digi\_root, digiReport\_digi\_root, svac\_digi\_root,
and reconReport\_digi\_root are all links to the same data which must appear
with all those names to work with GINO's naming convention.

There are more edges.  These represent extra information that must be
maintained by the user, as the ordering of TPs within a task must be
explicitly specified to GINO, and launching of other tasks by TPs and
launching of tasks are done ``by hand,'' without GINO's knowledge.

The dependency on the eLog DB is not explicitly present - it is guaranteed by
the ordering of the tasks.

I have in mind an alternate (harder, of course) approach to launching tasks
that would reduce latency waiting for tasks to run and simplify this graph
somewhat.  This is shown in {\figref{newlaunch-fig}}.


\begin{figure}
\caption{Proposed}

\includegraphics[bb=100 38 620 502,width=7.5in,height=6.5in]{newlaunch-EM2-v1r0.eps}

{The new scheme.}

\label{newlaunch-fig}
\end{figure}


\end{document}
